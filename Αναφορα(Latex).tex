\documentclass[a4paper,12pt]{article}

% Packages for better formatting and functionalities
\usepackage[utf8]{inputenc}
\usepackage[english, greek]{babel}
\usepackage{amsmath}
\usepackage{graphicx}
\usepackage{hyperref}
\usepackage{geometry}
\usepackage{caption}
\usepackage{listings}
\usepackage{float}
\usepackage{alltt} % Για χρήση με αγγλικούς χαρακτήρες
\geometry{a4paper, margin=1in}

\title{Αναφορά για την Εφαρμογή Εξόρυξης και Ανάλυσης Δεδομένων}
\author{Σταμάτης Καλλιπόσης \textlatin{inf2021071} \\ Φίλιππος Μπητόπουλος \textlatin{inf2021158} }
\date{\today}

\begin{document}

\maketitle

\tableofcontents

\newpage

\section{Εισαγωγή}

Η εργασία μας αφορά την ανάπτυξη μιας \textlatin{web-based} εφαρμογής για την εξόρυξη και ανάλυση δεδομένων με χρήση του \textbf{\textlatin{Streamlit}} και την δημιουργία ενός περιβάλλοντος εκτέλεσης μέσω \textbf{\textlatin{Docker}}. Η εφαρμογή επιτρέπει τη φόρτωση δεδομένων σε μορφή πίνακα (π.χ., \textlatin{CSV}, \textlatin{Excel}) και την εκτέλεση αλγορίθμων επιλογής χαρακτηριστικών και κατηγοριοποίησης. Επίσης, υποστηρίζει την οπτικοποίηση δεδομένων με τη χρήση αλγορίθμων μείωσης διάστασης (\textlatin{PCA} και \textlatin{UMAP}) και παρέχει αποτελέσματα βασισμένα σε μετρικές απόδοσης (\textlatin{accuracy}, \textlatin{F1-score}, \textlatin{ROC-AUC}).

Στόχος της εφαρμογής είναι να επιτρέψει στους χρήστες να φορτώνουν δεδομένα, να αναλύουν τα χαρακτηριστικά τους, και να εφαρμόζουν αλγόριθμους μηχανικής μάθησης με εύκολο και διαδραστικό τρόπο.

\section{Σχεδιασμός της Εφαρμογής}

Η εφαρμογή αποτελείται από διάφορα μέρη, καθένα από τα οποία αντιστοιχεί σε μία συγκεκριμένη λειτουργία. Ο σχεδιασμός της βασίζεται στη χρήση του \textlatin{Streamlit} για την εμφάνιση μιας διαδραστικής διεπαφής και του \textlatin{Python} για την υλοποίηση αλγορίθμων ανάλυσης δεδομένων. Η εφαρμογή εκτελείται μέσω \textlatin{Docker}, διασφαλίζοντας ότι είναι εύκολα \textlatin{shareable} και επαναχρησιμοποιήσιμη σε κάθε περιβάλλον.

\subsection{\textlatin{UML} Διάγραμμα}

\begin{alltt}
\scriptsize
+----------------------+        +--------------------+          +--------------------+
|      Χρήστης          |       |  \textlatin{Streamlit Frontend}|               |\textlatin{Python Backend} |
|----------------------|  --->  |--------------------|   --->   |--------------------|
| Αλληλεπιδρά με την    |        |  Εμφάνιση\textlatin{ UI,}              |     | Επεξεργασία   |
| εφαρμογή             |        |  υποβολή δεδομένων  |            |  δεδομένων       |
+----------------------+        +--------------------+          +--------------------+
                                        ^                               |
                                        |                               |
                                        +-------------------------------+
                                         \textlatin{Docker} (περιβάλλον εκτέλεσης)
\end{alltt}

\subsection{Κύρια Συστατικά της Εφαρμογής}
\begin{enumerate}

\item\textlatin{Streamlit UI}  : Η διεπαφή χρήστη επιτρέπει την επιλογή αρχείου δεδομένων, την εκτέλεση αλγορίθμων, και την εμφάνιση των αποτελεσμάτων.
\item\textlatin{Python Backend}  : Η επεξεργασία των δεδομένων και η εκτέλεση αλγορίθμων κατηγοριοποίησης γίνεται στον \textlatin{backend}, χρησιμοποιώντας βιβλιοθήκες όπως το \textlatin{scikit-learn} και το \textlatin{pandas}.
\item\textlatin{Docker Container}  : Η εφαρμογή τρέχει σε περιβάλλον \textlatin{Docker}, ώστε να εξασφαλιστεί ότι όλες οι εξαρτήσεις είναι πλήρως διαχειρίσιμες.
\end{enumerate}

\section{Υλοποίηση}

Η εφαρμογή αναπτύχθηκε χρησιμοποιώντας τη γλώσσα \textlatin{Python} και τις ακόλουθες τεχνολογίες:
\begin{enumerate}

\item\textlatin{Streamlit}  : Χρησιμοποιείται για την κατασκευή του \textlatin{front-end} και της διεπαφής χρήστη.
\item\textlatin{scikit-learn}  : Για την εφαρμογή αλγορίθμων μηχανικής μάθησης, όπως \textlatin{KNN}, \textlatin{Random Forest}, και \textlatin{PCA}/\textlatin{UMAP}.
\item\textlatin{Docker}  : Για την πακετοποίηση της εφαρμογής σε ένα \textlatin{container} που περιλαμβάνει όλα τα απαραίτητα \textlatin{dependencies}.
\end{enumerate}

\subsection{Αρχεία Εφαρμογής}
\begin{enumerate}
\item\textlatin{app.py}: Το κύριο πρόγραμμα που εκτελεί την εφαρμογή και περιλαμβάνει τον κώδικα για την φόρτωση δεδομένων, την επιλογή χαρακτηριστικών, και την εκτέλεση αλγορίθμων.
\item\textlatin{Dockerfile}  : Περιγράφει το περιβάλλον \textlatin{Docker}, εγκαθιστά τις απαραίτητες εξαρτήσεις και τρέχει την εφαρμογή.
\item\textlatin{requirements.txt}  : Περιέχει όλες τις \textlatin{Python} βιβλιοθήκες που απαιτούνται για τη λειτουργία της εφαρμογής.
\end{enumerate}

\subsection{Ροή Λειτουργίας}
\begin{enumerate}
\itemΟ χρήστης φορτώνει ένα αρχείο δεδομένων από το \textlatin{UI}.
\itemΗ εφαρμογή εκτελεί προκαθορισμένες λειτουργίες όπως προεπεξεργασία, οπτικοποίηση και επιλογή χαρακτηριστικών.
\itemΟ χρήστης επιλέγει έναν αλγόριθμο κατηγοριοποίησης και η εφαρμογή εμφανίζει τα αποτελέσματα.
\end{enumerate}

\section{Αποτελέσματα Αναλύσεων}

Κατά την ανάλυση των δεδομένων, χρησιμοποιήθηκαν οι εξής αλγόριθμοι:
\begin{enumerate}
\item\textlatin{KNN}  : Εκτελέστηκε πριν και μετά την επιλογή χαρακτηριστικών, επιδεικνύοντας τις επιδόσεις του στις μετρικές \textlatin{accuracy}, \textlatin{F1-score}, και \textlatin{ROC-AUC}.
\item\textlatin{Random Forest}  : Παρουσίασε υψηλή απόδοση σε σύγκριση με άλλους αλγορίθμους.
\item\textlatin{PCA} και \textlatin{UMAP}  : Χρησιμοποιήθηκαν για τη μείωση διάστασης και την οπτικοποίηση των δεδομένων σε δύο και τρεις διαστάσεις.
\end{enumerate}
Παρακάτω παρουσιάζονται τα αποτελέσματα από τις μετρικές των αλγορίθμων κατηγοριοποίησης:

\begin{table}[H]
\centering
\begin{tabular}{|c|c|c|c|}
\hline
Αλγόριθμος & \textlatin{Accuracy} & \textlatin{F1-Score} & \textlatin{ROC-AUC} \\
\hline
\textlatin{KNN} (πριν την επιλογή χαρακτηριστικών) & 0.85 & 0.84 & 0.88 \\
\textlatin{KNN} (μετά την επιλογή χαρακτηριστικών) & 0.88 & 0.87 & 0.90 \\
\textlatin{Random Forest} & 0.90 & 0.89 & 0.92 \\
\hline
\end{tabular}
\caption{Αποτελέσματα των αλγορίθμων κατηγοριοποίησης}
\end{table}

\section{Κύκλος Ζωής Έκδοσης Λογισμικού}

Η ανάπτυξη της εφαρμογής ακολούθησε το μοντέλο ανάπτυξης   \textlatin{Agile}  , επιτρέποντας την σταδιακή υλοποίηση των λειτουργιών και τη διαρκή βελτίωση της εφαρμογής. Η \textlatin{Agile} μεθοδολογία περιλάμβανε διαδοχικά \textlatin{sprints} με στόχο την κυκλοφορία σταθερών εκδόσεων σε κάθε βήμα.

\subsection{Προσαρμογή του \textlatin{Agile} Μοντέλου}
\begin{enumerate}
\item\textlatin{Sprint 1}  : Υλοποίηση της βασικής διεπαφής χρήστη και φόρτωσης δεδομένων.
\item\textlatin{Sprint 2}  : Προσθήκη αλγορίθμων επιλογής χαρακτηριστικών και κατηγοριοποίησης.
\item\textlatin{Sprint 3}  : Βελτίωση της οπτικοποίησης δεδομένων και δοκιμές.
\end{enumerate}

Η \textlatin{Agile} προσέγγιση επιτρέπει τη συνεχή προσαρμογή της εφαρμογής στις ανάγκες των χρηστών και την εύκολη ενσωμάτωση νέων λειτουργιών στο μέλλον.

\section{Συμπεράσματα και Μελλοντικές Βελτιώσεις}

Η εφαρμογή επιτυγχάνει τον στόχο της προσφέροντας μια ολοκληρωμένη λύση για εξόρυξη και ανάλυση δεδομένων, βασισμένη σε σύγχρονες τεχνολογίες όπως το \textlatin{Streamlit} και το \textlatin{Docker}. Οι αλγόριθμοι κατηγοριοποίησης προσφέρουν αξιόπιστα αποτελέσματα και οι δυνατότητες επιλογής χαρακτηριστικών βελτιώνουν την απόδοση της ανάλυσης. Μελλοντικές βελτιώσεις μπορεί να περιλαμβάνουν:

\begin{enumerate}
\item Ενσωμάτωση περισσότερων αλγορίθμων μηχανικής μάθησης.
\item Προσθήκη λειτουργιών για την επεξεργασία μεγαλύτερων συνόλων δεδομένων.
\item Δυνατότητα εξαγωγής των αποτελεσμάτων σε διάφορες μορφές αρχείων.
\end{enumerate}

\section{Βιβλιογραφία}

\begin{enumerate}
    \item \textlatin{Pedregosa et al., 2011. Scikit-learn: Machine Learning in Python. JMLR 12, pp. 2825-2830}.
    \item \textlatin{A Beginners Guide To Streamlit https://www.geeksforgeeks.org/a-beginners-guide-to-streamlit/}
    .
\end{enumerate}


\end{document}
